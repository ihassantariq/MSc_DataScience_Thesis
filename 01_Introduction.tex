\chapter{Introduction} \label{chap1}



% Review for introduction: 
% - https://www.nature.com/articles/s41746-020-00323-1


\section{Document's outline} \label{doc_struct}

The present thesis document is organized into six chapters. 

The first chapter (\ref{chap1}) -- The second chapter (\ref{chap2}) describes the essential background to understand the cardiological basics. It contains the human heart nature and all the concepts associated with ECG monitoring, including the most common arrhythmia types.

The third chapter (\ref{chap3}) focuses on the state-of-the-art regarding the research topic. It contains information for both ECG classification -- . In the fourth chapter (\ref{chap4}), the analytical methodologies and tools are introduced and explained in detail. 

The fifth chapter (\ref{chap5}) it is exposed the results of analyzing the proposed data under  ---- Finally, the last chapter (\ref{chap6}) is dedicated to the overall conclusions achieved by the research and to possible future developments.