\chapter{Introduction} \label{chap1}

The Fourth Industrial Revolution is fostering the emergence of new scenarios in which vast volumes of data are shared among independent and potentially disparate organizations. Often used on a cross-border basis to improve shared services in several sectors, such as finance and health care. Despite its benefits, technological advancements are introducing new security and privacy concerns associated with the use of these data, which include factors such as collection, analysis, usage, storage, and sharing. Indeed, in the case of personal information, incorrect usage, unsafe storage, data leakage, or misuse can all compromise a person's privacy. As a result, when personal data is subject to federated computation, the availability and proper use of privacy-preserving and fairness-aware mechanisms are presented as a key element to be addressed to increase people's trust and thus achieving the sustainable and ethical realization of these scenarios.

Digital Health Products (DHP) in the eHealth sector, in particular, present unique options to provide efficient, effective, cross-border high-quality healthcare services ~\cite{world2017global}. Today, cutting-edge AI-based medical data analysis has promise for early detection, faster diagnosis, better decision-making, and more successful treatment, according to ~\cite{arnold2017doctor}. The use of AI-based DHP in healthcare operations, services, and applications has created a significant and pressing need to combine highly private medical data gathered from a variety of sources. It also includes millions of parameters that must be learned from sufficiently big, curated datasets to reach clinical-grade accuracy while remaining safe, fair, and equitable, as well as generalizing well to previously unseen ~\cite{wang2019deep}.

Federated learning (FL) is an architecture that aims to solve the problem of data governance and privacy by collectively training algorithms without transferring data. It was originally designed for a variety of domains, including mobile and edge device use cases, but it has recently acquired popularity in healthcare applications \cite{intro}. FL allows for collaborative insights, such as in the form of a consensus model, without transferring patient data outside of the institutions' firewalls. Instead, each participating institution's ML process takes place locally, with only model weights being shared. Models trained by FL can achieve performance levels comparable to those trained on centrally hosted data sets and superior to models that only see isolated single-institutional data, according to recent studies \cite{intro}.

% Review for introduction: 
% - https://www.nature.com/articles/s41746-020-00323-1


\section{Document's outline} \label{doc_struct}

The present thesis document is organized into six chapters. 

The first chapter (\ref{chap1}) introduces the reasons behind the need for a machine learning solution to classify the 12-leads, encouraging the introduction of Federated Learning in the architecture. The second chapter (\ref{chap2}) describes the essential background to understand the cardiological basics. It contains the human heart nature and all the concepts associated with ECG monitoring, including the most common arrhythmia types.

The third chapter (\ref{chap3}) focuses on the state-of-the-art regarding the research topic. It contains information for both ECG classification and Federated Learning. In the fourth chapter (\ref{chap4}), the analytical methodologies and tools are introduced and explained in detail. 

The fifth chapter (\ref{chap5}) it is exposed the results of analyzing the proposed data under both the Centralized and Federated Learning environment. Finally, the last chapter (\ref{chap6}) is dedicated to the overall conclusions achieved by the research and to possible future developments.